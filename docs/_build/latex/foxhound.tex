%% Generated by Sphinx.
\def\sphinxdocclass{report}
\documentclass[letterpaper,10pt,english]{sphinxmanual}
\ifdefined\pdfpxdimen
   \let\sphinxpxdimen\pdfpxdimen\else\newdimen\sphinxpxdimen
\fi \sphinxpxdimen=.75bp\relax
\ifdefined\pdfimageresolution
    \pdfimageresolution= \numexpr \dimexpr1in\relax/\sphinxpxdimen\relax
\fi
%% let collapsible pdf bookmarks panel have high depth per default
\PassOptionsToPackage{bookmarksdepth=5}{hyperref}

\PassOptionsToPackage{warn}{textcomp}
\usepackage[utf8]{inputenc}
\ifdefined\DeclareUnicodeCharacter
% support both utf8 and utf8x syntaxes
  \ifdefined\DeclareUnicodeCharacterAsOptional
    \def\sphinxDUC#1{\DeclareUnicodeCharacter{"#1}}
  \else
    \let\sphinxDUC\DeclareUnicodeCharacter
  \fi
  \sphinxDUC{00A0}{\nobreakspace}
  \sphinxDUC{2500}{\sphinxunichar{2500}}
  \sphinxDUC{2502}{\sphinxunichar{2502}}
  \sphinxDUC{2514}{\sphinxunichar{2514}}
  \sphinxDUC{251C}{\sphinxunichar{251C}}
  \sphinxDUC{2572}{\textbackslash}
\fi
\usepackage{cmap}
\usepackage[T1]{fontenc}
\usepackage{amsmath,amssymb,amstext}
\usepackage{babel}



\usepackage{tgtermes}
\usepackage{tgheros}
\renewcommand{\ttdefault}{txtt}



\usepackage[Bjarne]{fncychap}
\usepackage{sphinx}

\fvset{fontsize=auto}
\usepackage{geometry}


% Include hyperref last.
\usepackage{hyperref}
% Fix anchor placement for figures with captions.
\usepackage{hypcap}% it must be loaded after hyperref.
% Set up styles of URL: it should be placed after hyperref.
\urlstyle{same}


\usepackage{sphinxmessages}
\setcounter{tocdepth}{2}



\title{Foxhound}
\date{Oct 21, 2021}
\release{0.1.2}
\author{João Gabriel Segato Kruse}
\newcommand{\sphinxlogo}{\vbox{}}
\renewcommand{\releasename}{Release}
\makeindex
\begin{document}

\pagestyle{empty}
\sphinxmaketitle
\pagestyle{plain}
\sphinxtableofcontents
\pagestyle{normal}
\phantomsection\label{\detokenize{index::doc}}



\chapter{Application Logic  Module}
\label{\detokenize{application_logic:module-application_logic}}\label{\detokenize{application_logic:application-logic-module}}\label{\detokenize{application_logic::doc}}\index{module@\spxentry{module}!application\_logic@\spxentry{application\_logic}}\index{application\_logic@\spxentry{application\_logic}!module@\spxentry{module}}\index{App (class in application\_logic)@\spxentry{App}\spxextra{class in application\_logic}}

\begin{fulllineitems}
\phantomsection\label{\detokenize{application_logic:application_logic.App}}\pysiglinewithargsret{\sphinxbfcode{\sphinxupquote{class }}\sphinxcode{\sphinxupquote{application\_logic.}}\sphinxbfcode{\sphinxupquote{App}}}{\emph{\DUrole{n}{name}\DUrole{o}{=}\DUrole{default_value}{\textquotesingle{}Foxhound\textquotesingle{}}}, \emph{\DUrole{n}{img}\DUrole{o}{=}\DUrole{default_value}{\textquotesingle{}Imgs/foxhound.ico\textquotesingle{}}}, \emph{\DUrole{n}{config\_img}\DUrole{o}{=}\DUrole{default_value}{\textquotesingle{}Imgs/config.ico\textquotesingle{}}}}{}
\sphinxAtStartPar
Class used to control the operations of the application

\end{fulllineitems}

\index{Toolbar (class in application\_logic)@\spxentry{Toolbar}\spxextra{class in application\_logic}}

\begin{fulllineitems}
\phantomsection\label{\detokenize{application_logic:application_logic.Toolbar}}\pysiglinewithargsret{\sphinxbfcode{\sphinxupquote{class }}\sphinxcode{\sphinxupquote{application\_logic.}}\sphinxbfcode{\sphinxupquote{Toolbar}}}{\emph{\DUrole{o}{*}\DUrole{n}{args}}, \emph{\DUrole{o}{**}\DUrole{n}{kwargs}}}{}
\end{fulllineitems}



\chapter{Interface  Module}
\label{\detokenize{interface:module-interface}}\label{\detokenize{interface:interface-module}}\label{\detokenize{interface::doc}}\index{module@\spxentry{module}!interface@\spxentry{interface}}\index{interface@\spxentry{interface}!module@\spxentry{module}}\index{Interface (class in interface)@\spxentry{Interface}\spxextra{class in interface}}

\begin{fulllineitems}
\phantomsection\label{\detokenize{interface:interface.Interface}}\pysigline{\sphinxbfcode{\sphinxupquote{class }}\sphinxcode{\sphinxupquote{interface.}}\sphinxbfcode{\sphinxupquote{Interface}}}
\sphinxAtStartPar
Class responsible for all interfaces with the UI,
managing the opened windows and reading and writting
to the UI.
\index{create\_tree() (interface.Interface method)@\spxentry{create\_tree()}\spxextra{interface.Interface method}}

\begin{fulllineitems}
\phantomsection\label{\detokenize{interface:interface.Interface.create_tree}}\pysiglinewithargsret{\sphinxbfcode{\sphinxupquote{create\_tree}}}{\emph{\DUrole{n}{values}}, \emph{\DUrole{n}{index}\DUrole{o}{=}\DUrole{default_value}{None}}}{}
\sphinxAtStartPar
Create tree structure to display
\begin{quote}\begin{description}
\item[{Parameters}] \leavevmode\begin{itemize}
\item {} 
\sphinxAtStartPar
\sphinxstyleliteralstrong{\sphinxupquote{values}} (\sphinxstyleliteralemphasis{\sphinxupquote{iterable}}) \textendash{} iterable with the element for each position on the tree

\item {} 
\sphinxAtStartPar
\sphinxstyleliteralstrong{\sphinxupquote{index}} (\sphinxstyleliteralemphasis{\sphinxupquote{iterable}}\sphinxstyleliteralemphasis{\sphinxupquote{, }}\sphinxstyleliteralemphasis{\sphinxupquote{optional}}) \textendash{} Indexes to use instead of the position of the element in values

\end{itemize}

\item[{Returns}] \leavevmode
\sphinxAtStartPar
Tree with the given data

\item[{Return type}] \leavevmode
\sphinxAtStartPar
sg.Tree

\end{description}\end{quote}

\end{fulllineitems}

\index{create\_window() (interface.Interface method)@\spxentry{create\_window()}\spxextra{interface.Interface method}}

\begin{fulllineitems}
\phantomsection\label{\detokenize{interface:interface.Interface.create_window}}\pysiglinewithargsret{\sphinxbfcode{\sphinxupquote{create\_window}}}{\emph{\DUrole{n}{name}}, \emph{\DUrole{n}{layout}}, \emph{\DUrole{o}{**}\DUrole{n}{opt}}}{}
\sphinxAtStartPar
Create window with the given specifications
\begin{quote}\begin{description}
\item[{Parameters}] \leavevmode\begin{itemize}
\item {} 
\sphinxAtStartPar
\sphinxstyleliteralstrong{\sphinxupquote{name}} (\sphinxstyleliteralemphasis{\sphinxupquote{str}}) \textendash{} Window name

\item {} 
\sphinxAtStartPar
\sphinxstyleliteralstrong{\sphinxupquote{layout}} (\sphinxstyleliteralemphasis{\sphinxupquote{List}}\sphinxstyleliteralemphasis{\sphinxupquote{{[}}}\sphinxstyleliteralemphasis{\sphinxupquote{List}}\sphinxstyleliteralemphasis{\sphinxupquote{{[}}}\sphinxstyleliteralemphasis{\sphinxupquote{Element}}\sphinxstyleliteralemphasis{\sphinxupquote{{]}}}\sphinxstyleliteralemphasis{\sphinxupquote{{]}}}) \textendash{} Layout for the window

\item {} 
\sphinxAtStartPar
\sphinxstyleliteralstrong{\sphinxupquote{**opt}} (\sphinxstyleliteralemphasis{\sphinxupquote{kwargs}}) \textendash{} Optional parameters (icon, resizable, maximizer, etc)

\end{itemize}

\item[{Returns}] \leavevmode
\sphinxAtStartPar
Window identifier for reading and writting from and to this specific window

\item[{Return type}] \leavevmode
\sphinxAtStartPar
int

\end{description}\end{quote}

\end{fulllineitems}

\index{get\_selected\_row() (interface.Interface method)@\spxentry{get\_selected\_row()}\spxextra{interface.Interface method}}

\begin{fulllineitems}
\phantomsection\label{\detokenize{interface:interface.Interface.get_selected_row}}\pysiglinewithargsret{\sphinxbfcode{\sphinxupquote{get\_selected\_row}}}{\emph{\DUrole{n}{element}}, \emph{\DUrole{n}{index}\DUrole{o}{=}\DUrole{default_value}{0}}}{}
\sphinxAtStartPar
Get selected row from tree
\begin{quote}\begin{description}
\item[{Parameters}] \leavevmode\begin{itemize}
\item {} 
\sphinxAtStartPar
\sphinxstyleliteralstrong{\sphinxupquote{element}} (\sphinxstyleliteralemphasis{\sphinxupquote{str}}) \textendash{} Key for the tree element on the window

\item {} 
\sphinxAtStartPar
\sphinxstyleliteralstrong{\sphinxupquote{index}} (\sphinxstyleliteralemphasis{\sphinxupquote{int}}\sphinxstyleliteralemphasis{\sphinxupquote{, }}\sphinxstyleliteralemphasis{\sphinxupquote{optional}}) \textendash{} Identifier for which window to get from

\end{itemize}

\end{description}\end{quote}

\end{fulllineitems}

\index{get\_window() (interface.Interface method)@\spxentry{get\_window()}\spxextra{interface.Interface method}}

\begin{fulllineitems}
\phantomsection\label{\detokenize{interface:interface.Interface.get_window}}\pysiglinewithargsret{\sphinxbfcode{\sphinxupquote{get\_window}}}{}{}
\sphinxAtStartPar
Get window
\begin{quote}\begin{description}
\item[{Parameters}] \leavevmode
\sphinxAtStartPar
\sphinxstyleliteralstrong{\sphinxupquote{index}} (\sphinxstyleliteralemphasis{\sphinxupquote{int}}\sphinxstyleliteralemphasis{\sphinxupquote{, }}\sphinxstyleliteralemphasis{\sphinxupquote{optional}}) \textendash{} Window identifier (default is 0, meaning the first window created)

\item[{Returns}] \leavevmode
\sphinxAtStartPar
Window object

\item[{Return type}] \leavevmode
\sphinxAtStartPar
sg.Window

\end{description}\end{quote}

\end{fulllineitems}

\index{popup() (interface.Interface method)@\spxentry{popup()}\spxextra{interface.Interface method}}

\begin{fulllineitems}
\phantomsection\label{\detokenize{interface:interface.Interface.popup}}\pysiglinewithargsret{\sphinxbfcode{\sphinxupquote{popup}}}{\emph{\DUrole{n}{message}}}{}
\sphinxAtStartPar
Display popup with message
\begin{quote}\begin{description}
\item[{Parameters}] \leavevmode
\sphinxAtStartPar
\sphinxstyleliteralstrong{\sphinxupquote{message}} (\sphinxstyleliteralemphasis{\sphinxupquote{str}}) \textendash{} Message to be displayed

\end{description}\end{quote}

\end{fulllineitems}

\index{read\_events() (interface.Interface method)@\spxentry{read\_events()}\spxextra{interface.Interface method}}

\begin{fulllineitems}
\phantomsection\label{\detokenize{interface:interface.Interface.read_events}}\pysiglinewithargsret{\sphinxbfcode{\sphinxupquote{read\_events}}}{\emph{\DUrole{n}{timeout}\DUrole{o}{=}\DUrole{default_value}{None}}, \emph{\DUrole{n}{index}\DUrole{o}{=}\DUrole{default_value}{0}}}{}
\sphinxAtStartPar
Read events and values from window
\begin{quote}\begin{description}
\item[{Parameters}] \leavevmode\begin{itemize}
\item {} 
\sphinxAtStartPar
\sphinxstyleliteralstrong{\sphinxupquote{timeout}} (\sphinxstyleliteralemphasis{\sphinxupquote{int}}\sphinxstyleliteralemphasis{\sphinxupquote{, }}\sphinxstyleliteralemphasis{\sphinxupquote{optional}}) \textendash{} Timeout to wait for event in ms (default: None)

\item {} 
\sphinxAtStartPar
\sphinxstyleliteralstrong{\sphinxupquote{index}} (\sphinxstyleliteralemphasis{\sphinxupquote{index}}) \textendash{} Identifier for the target window (default: 0)

\end{itemize}

\item[{Returns}] \leavevmode
\sphinxAtStartPar
Events and values of the given window

\item[{Return type}] \leavevmode
\sphinxAtStartPar
(events, values)

\end{description}\end{quote}

\end{fulllineitems}

\index{start\_loading() (interface.Interface method)@\spxentry{start\_loading()}\spxextra{interface.Interface method}}

\begin{fulllineitems}
\phantomsection\label{\detokenize{interface:interface.Interface.start_loading}}\pysiglinewithargsret{\sphinxbfcode{\sphinxupquote{start\_loading}}}{\emph{\DUrole{n}{timeout}\DUrole{o}{=}\DUrole{default_value}{100}}}{}
\sphinxAtStartPar
Start displaying loading screen
\begin{quote}\begin{description}
\item[{Parameters}] \leavevmode
\sphinxAtStartPar
\sphinxstyleliteralstrong{\sphinxupquote{timeout}} (\sphinxstyleliteralemphasis{\sphinxupquote{int}}) \textendash{} Timeout between frames in ms

\end{description}\end{quote}

\end{fulllineitems}

\index{stop\_loading() (interface.Interface method)@\spxentry{stop\_loading()}\spxextra{interface.Interface method}}

\begin{fulllineitems}
\phantomsection\label{\detokenize{interface:interface.Interface.stop_loading}}\pysiglinewithargsret{\sphinxbfcode{\sphinxupquote{stop\_loading}}}{}{}
\sphinxAtStartPar
Stop loading animation

\end{fulllineitems}

\index{update\_element() (interface.Interface method)@\spxentry{update\_element()}\spxextra{interface.Interface method}}

\begin{fulllineitems}
\phantomsection\label{\detokenize{interface:interface.Interface.update_element}}\pysiglinewithargsret{\sphinxbfcode{\sphinxupquote{update\_element}}}{\emph{\DUrole{n}{data}}, \emph{\DUrole{n}{element}}, \emph{\DUrole{n}{arg\_name}\DUrole{o}{=}\DUrole{default_value}{\textquotesingle{}value\textquotesingle{}}}, \emph{\DUrole{n}{index}\DUrole{o}{=}\DUrole{default_value}{0}}}{}
\sphinxAtStartPar
Updates the element on the screen
\begin{quote}\begin{description}
\item[{Parameters}] \leavevmode\begin{itemize}
\item {} 
\sphinxAtStartPar
\sphinxstyleliteralstrong{\sphinxupquote{data}} (\sphinxstyleliteralemphasis{\sphinxupquote{Object}}) \textendash{} Data to be updated

\item {} 
\sphinxAtStartPar
\sphinxstyleliteralstrong{\sphinxupquote{element}} (\sphinxstyleliteralemphasis{\sphinxupquote{str}}) \textendash{} Key for the element on the window

\item {} 
\sphinxAtStartPar
\sphinxstyleliteralstrong{\sphinxupquote{arg\_name}} (\sphinxstyleliteralemphasis{\sphinxupquote{str}}\sphinxstyleliteralemphasis{\sphinxupquote{, }}\sphinxstyleliteralemphasis{\sphinxupquote{optional}}) \textendash{} Name of the argument in the Update method

\item {} 
\sphinxAtStartPar
\sphinxstyleliteralstrong{\sphinxupquote{index}} (\sphinxstyleliteralemphasis{\sphinxupquote{int}}\sphinxstyleliteralemphasis{\sphinxupquote{, }}\sphinxstyleliteralemphasis{\sphinxupquote{optional}}) \textendash{} Identifier for which window to update

\end{itemize}

\end{description}\end{quote}

\end{fulllineitems}

\index{update\_tree() (interface.Interface method)@\spxentry{update\_tree()}\spxextra{interface.Interface method}}

\begin{fulllineitems}
\phantomsection\label{\detokenize{interface:interface.Interface.update_tree}}\pysiglinewithargsret{\sphinxbfcode{\sphinxupquote{update\_tree}}}{\emph{\DUrole{n}{data}}, \emph{\DUrole{n}{element}}, \emph{\DUrole{n}{index}\DUrole{o}{=}\DUrole{default_value}{0}}}{}
\sphinxAtStartPar
Updates the tree on the screen
\begin{quote}\begin{description}
\item[{Parameters}] \leavevmode\begin{itemize}
\item {} 
\sphinxAtStartPar
\sphinxstyleliteralstrong{\sphinxupquote{data}} (\sphinxstyleliteralemphasis{\sphinxupquote{iterable}}) \textendash{} Data representing the tree

\item {} 
\sphinxAtStartPar
\sphinxstyleliteralstrong{\sphinxupquote{element}} (\sphinxstyleliteralemphasis{\sphinxupquote{str}}) \textendash{} Key for the tree element on the window

\item {} 
\sphinxAtStartPar
\sphinxstyleliteralstrong{\sphinxupquote{index}} (\sphinxstyleliteralemphasis{\sphinxupquote{int}}\sphinxstyleliteralemphasis{\sphinxupquote{, }}\sphinxstyleliteralemphasis{\sphinxupquote{optional}}) \textendash{} Identifier for which window to update

\end{itemize}

\end{description}\end{quote}

\end{fulllineitems}

\index{write\_event() (interface.Interface method)@\spxentry{write\_event()}\spxextra{interface.Interface method}}

\begin{fulllineitems}
\phantomsection\label{\detokenize{interface:interface.Interface.write_event}}\pysiglinewithargsret{\sphinxbfcode{\sphinxupquote{write\_event}}}{\emph{\DUrole{n}{name}}, \emph{\DUrole{n}{params}}, \emph{\DUrole{n}{index}\DUrole{o}{=}\DUrole{default_value}{0}}}{}
\sphinxAtStartPar
Get selected row from tree
\begin{quote}\begin{description}
\item[{Parameters}] \leavevmode\begin{itemize}
\item {} 
\sphinxAtStartPar
\sphinxstyleliteralstrong{\sphinxupquote{name}} (\sphinxstyleliteralemphasis{\sphinxupquote{str}}) \textendash{} Key for the event name

\item {} 
\sphinxAtStartPar
\sphinxstyleliteralstrong{\sphinxupquote{params}} (\sphinxstyleliteralemphasis{\sphinxupquote{Object}}) \textendash{} Parameters to pass to the event

\item {} 
\sphinxAtStartPar
\sphinxstyleliteralstrong{\sphinxupquote{index}} (\sphinxstyleliteralemphasis{\sphinxupquote{int}}\sphinxstyleliteralemphasis{\sphinxupquote{, }}\sphinxstyleliteralemphasis{\sphinxupquote{optional}}) \textendash{} Identifier for which window to create from

\end{itemize}

\end{description}\end{quote}

\end{fulllineitems}


\end{fulllineitems}



\chapter{Causations  Module}
\label{\detokenize{causations:module-causations}}\label{\detokenize{causations:causations-module}}\label{\detokenize{causations::doc}}\index{module@\spxentry{module}!causations@\spxentry{causations}}\index{causations@\spxentry{causations}!module@\spxentry{module}}\index{Causations (class in causations)@\spxentry{Causations}\spxextra{class in causations}}

\begin{fulllineitems}
\phantomsection\label{\detokenize{causations:causations.Causations}}\pysiglinewithargsret{\sphinxbfcode{\sphinxupquote{class }}\sphinxcode{\sphinxupquote{causations.}}\sphinxbfcode{\sphinxupquote{Causations}}}{\emph{\DUrole{o}{**}\DUrole{n}{opt}}}{}
\sphinxAtStartPar
Class responsible for causation finding with TCDF
\index{get\_causation() (causations.Causations method)@\spxentry{get\_causation()}\spxextra{causations.Causations method}}

\begin{fulllineitems}
\phantomsection\label{\detokenize{causations:causations.Causations.get_causation}}\pysiglinewithargsret{\sphinxbfcode{\sphinxupquote{get\_causation}}}{\emph{\DUrole{n}{datafiles}}}{}
\sphinxAtStartPar
Finds causations between all variables in datafiles
\begin{quote}\begin{description}
\item[{Parameters}] \leavevmode
\sphinxAtStartPar
\sphinxstyleliteralstrong{\sphinxupquote{alldelays}} (\sphinxtitleref{pandas.DataFrame}) \textendash{} DataFrame containing all time series to be considered

\item[{Returns}] \leavevmode
\sphinxAtStartPar
All delays and variable names

\item[{Return type}] \leavevmode
\sphinxAtStartPar
\sphinxtitleref{(List{[}List{[}int{]}{]}, List{[}str{]})}

\end{description}\end{quote}

\end{fulllineitems}

\index{getextendeddelays() (causations.Causations method)@\spxentry{getextendeddelays()}\spxextra{causations.Causations method}}

\begin{fulllineitems}
\phantomsection\label{\detokenize{causations:causations.Causations.getextendeddelays}}\pysiglinewithargsret{\sphinxbfcode{\sphinxupquote{getextendeddelays}}}{\emph{\DUrole{n}{gtfile}}, \emph{\DUrole{n}{columns}}}{}
\sphinxAtStartPar
Collects the total delay of indirect causal relationships.

\end{fulllineitems}

\index{plotgraph() (causations.Causations static method)@\spxentry{plotgraph()}\spxextra{causations.Causations static method}}

\begin{fulllineitems}
\phantomsection\label{\detokenize{causations:causations.Causations.plotgraph}}\pysiglinewithargsret{\sphinxbfcode{\sphinxupquote{static }}\sphinxbfcode{\sphinxupquote{plotgraph}}}{\emph{\DUrole{n}{alldelays}}, \emph{\DUrole{n}{columns}}}{}
\sphinxAtStartPar
Plots a temporal causal graph showing all discovered causal relationships annotated with the time delay between cause and effect.
\begin{quote}\begin{description}
\item[{Parameters}] \leavevmode\begin{itemize}
\item {} 
\sphinxAtStartPar
\sphinxstyleliteralstrong{\sphinxupquote{alldelays}} (\sphinxtitleref{List{[}List{[}int{]}{]}}) \textendash{} delays between each two variables

\item {} 
\sphinxAtStartPar
\sphinxstyleliteralstrong{\sphinxupquote{columns}} (\sphinxtitleref{List{[}str{]}}) \textendash{} List with all variable names

\end{itemize}

\end{description}\end{quote}

\end{fulllineitems}

\index{runTCDF() (causations.Causations method)@\spxentry{runTCDF()}\spxextra{causations.Causations method}}

\begin{fulllineitems}
\phantomsection\label{\detokenize{causations:causations.Causations.runTCDF}}\pysiglinewithargsret{\sphinxbfcode{\sphinxupquote{runTCDF}}}{\emph{\DUrole{n}{df\_data}}}{}
\sphinxAtStartPar
Loops through all variables in a dataset and return the discovered causes, time delays, losses, attention scores and variable names.

\end{fulllineitems}


\end{fulllineitems}



\chapter{Correlator  Module}
\label{\detokenize{correlator:module-correlator}}\label{\detokenize{correlator:correlator-module}}\label{\detokenize{correlator::doc}}\index{module@\spxentry{module}!correlator@\spxentry{correlator}}\index{correlator@\spxentry{correlator}!module@\spxentry{module}}\index{correlate() (in module correlator)@\spxentry{correlate()}\spxextra{in module correlator}}

\begin{fulllineitems}
\phantomsection\label{\detokenize{correlator:correlator.correlate}}\pysiglinewithargsret{\sphinxcode{\sphinxupquote{correlator.}}\sphinxbfcode{\sphinxupquote{correlate}}}{\emph{\DUrole{n}{x}}, \emph{\DUrole{n}{y}}, \emph{\DUrole{n}{margin}}, \emph{\DUrole{n}{method}\DUrole{o}{=}\DUrole{default_value}{\textquotesingle{}pearson\textquotesingle{}}}}{}
\sphinxAtStartPar
Find delay and correlation between x and each column o y
\begin{quote}\begin{description}
\item[{Parameters}] \leavevmode\begin{itemize}
\item {} 
\sphinxAtStartPar
\sphinxstyleliteralstrong{\sphinxupquote{x}} (\sphinxtitleref{pandas.Series}) \textendash{} Main signal

\item {} 
\sphinxAtStartPar
\sphinxstyleliteralstrong{\sphinxupquote{y}} (\sphinxtitleref{pandas.DataFrame}) \textendash{} Secondary signals

\item {} 
\sphinxAtStartPar
\sphinxstyleliteralstrong{\sphinxupquote{method}} (\sphinxtitleref{str}, optional) \textendash{} Correlation method. Defaults to \sphinxtitleref{pearson}. Options: \sphinxtitleref{pearson},\textasciigrave{}robust\textasciigrave{},\textasciigrave{}kendall\textasciigrave{},\textasciigrave{}spearman\textasciigrave{}

\end{itemize}

\item[{Returns}] \leavevmode
\sphinxAtStartPar
List of correlation coefficients and delays in samples in the same order as y’s columns

\item[{Return type}] \leavevmode
\sphinxAtStartPar
\sphinxtitleref{(List{[}float{]}, List{[}int{]})}

\end{description}\end{quote}
\subsubsection*{Notes}

\sphinxAtStartPar
Uses the pandas method corrwith (which can return pearson, kendall or spearman coefficients) to correlate. If robust
correlation is used, the mapping presented in \sphinxstepexplicit %
\begin{footnote}[1]\phantomsection\label{\thesphinxscope.1}%
\sphinxAtStartFootnote
Raymaekers, J., Rousseeuw, P. “Fast Robust Correlation for High\sphinxhyphen{}Dimensional Data”, Technometrics, vol. 63, Pages 184\sphinxhyphen{}198, 2021
%
\end{footnote} is used and then Pearson correlation is used. To speedup the lag finding,
the delays are calculated in log intervals and then interpolated by splines, as shown in \sphinxstepexplicit %
\begin{footnote}[2]\phantomsection\label{\thesphinxscope.2}%
\sphinxAtStartFootnote
Sakurai, Yasushi \& Papadimitriou, Spiros \& Faloutsos, Christos. (2005). BRAID: Stream mining through group lag correlations. Proceedings of the ACM SIGMOD International Conference on Management of Data. 599\sphinxhyphen{}610.
%
\end{footnote}, and the lag with maximum correlation
found in this interpolated function is then used as the delay.
\subsubsection*{References}

\end{fulllineitems}

\index{find\_delays() (in module correlator)@\spxentry{find\_delays()}\spxextra{in module correlator}}

\begin{fulllineitems}
\phantomsection\label{\detokenize{correlator:correlator.find_delays}}\pysiglinewithargsret{\sphinxcode{\sphinxupquote{correlator.}}\sphinxbfcode{\sphinxupquote{find\_delays}}}{\emph{\DUrole{n}{x}}, \emph{\DUrole{n}{y}}}{}
\sphinxAtStartPar
Find delay between x and each column o y
\begin{quote}\begin{description}
\item[{Parameters}] \leavevmode\begin{itemize}
\item {} 
\sphinxAtStartPar
\sphinxstyleliteralstrong{\sphinxupquote{x}} (\sphinxtitleref{pandas.Series}) \textendash{} Main signal

\item {} 
\sphinxAtStartPar
\sphinxstyleliteralstrong{\sphinxupquote{y}} (\sphinxtitleref{pandas.DataFrame}) \textendash{} Secondary signals

\end{itemize}

\item[{Returns}] \leavevmode
\sphinxAtStartPar
Dataframe with the delay value for each column of y

\item[{Return type}] \leavevmode
\sphinxAtStartPar
\sphinxtitleref{pandas.DataFrame}

\end{description}\end{quote}

\end{fulllineitems}

\index{interpolate() (in module correlator)@\spxentry{interpolate()}\spxextra{in module correlator}}

\begin{fulllineitems}
\phantomsection\label{\detokenize{correlator:correlator.interpolate}}\pysiglinewithargsret{\sphinxcode{\sphinxupquote{correlator.}}\sphinxbfcode{\sphinxupquote{interpolate}}}{\emph{\DUrole{n}{x}}, \emph{\DUrole{n}{idx}}, \emph{\DUrole{n}{margin}}}{}
\sphinxAtStartPar
Interpolate data to match idx+\sphinxhyphen{}margin
\begin{quote}\begin{description}
\item[{Parameters}] \leavevmode\begin{itemize}
\item {} 
\sphinxAtStartPar
\sphinxstyleliteralstrong{\sphinxupquote{x}} (\sphinxtitleref{pandas.Dataframe}) \textendash{} Signal

\item {} 
\sphinxAtStartPar
\sphinxstyleliteralstrong{\sphinxupquote{idx}} (\sphinxtitleref{pandas.DatetimeIndex}) \textendash{} Index to match

\item {} 
\sphinxAtStartPar
\sphinxstyleliteralstrong{\sphinxupquote{margin}} (\sphinxtitleref{float}) \textendash{} Percentage of values to add to each side of index

\end{itemize}

\item[{Returns}] \leavevmode
\sphinxAtStartPar
Dataframe with the same columns as x interpolated to match idx+\sphinxhyphen{}margin

\item[{Return type}] \leavevmode
\sphinxAtStartPar
\sphinxtitleref{pandas.DataFrame}

\end{description}\end{quote}
\subsubsection*{Notes}

\sphinxAtStartPar
It infers the frequency for the given DatetimeIndex and extends it to margin times prior
and after. This new DatetimeIndex is then combined with the given DataFrame and the NaN
values are completed with linear interpolation then. In the end, only the new index values
are kept, so that it matches exactly the given idx dates (except for the margin values).

\end{fulllineitems}

\index{lagged\_corr() (in module correlator)@\spxentry{lagged\_corr()}\spxextra{in module correlator}}

\begin{fulllineitems}
\phantomsection\label{\detokenize{correlator:correlator.lagged_corr}}\pysiglinewithargsret{\sphinxcode{\sphinxupquote{correlator.}}\sphinxbfcode{\sphinxupquote{lagged\_corr}}}{\emph{\DUrole{n}{x}}, \emph{\DUrole{n}{y}}, \emph{\DUrole{n}{lag}}, \emph{\DUrole{n}{method}\DUrole{o}{=}\DUrole{default_value}{\textquotesingle{}pearson\textquotesingle{}}}}{}
\sphinxAtStartPar
Find correlation between x and each column o y for a specific time lag
\begin{quote}\begin{description}
\item[{Parameters}] \leavevmode\begin{itemize}
\item {} 
\sphinxAtStartPar
\sphinxstyleliteralstrong{\sphinxupquote{x}} (\sphinxtitleref{pandas.Series}) \textendash{} Main signal

\item {} 
\sphinxAtStartPar
\sphinxstyleliteralstrong{\sphinxupquote{y}} (\sphinxtitleref{pandas.DataFrame}) \textendash{} Secondary signals

\item {} 
\sphinxAtStartPar
\sphinxstyleliteralstrong{\sphinxupquote{lag}} (\sphinxtitleref{int}) \textendash{} Number of samples to apply as lag before computing the correlation

\item {} 
\sphinxAtStartPar
\sphinxstyleliteralstrong{\sphinxupquote{method}} (\sphinxtitleref{str}, optional) \textendash{} Correlation method. Defaults to \sphinxtitleref{pearson}. Options: \sphinxtitleref{pearson},\textasciigrave{}kendall\textasciigrave{},\textasciigrave{}spearman\textasciigrave{}

\end{itemize}

\item[{Returns}] \leavevmode
\sphinxAtStartPar
Dataframe with the correlation value for each column of y

\item[{Return type}] \leavevmode
\sphinxAtStartPar
\sphinxtitleref{pandas.DataFrame}

\end{description}\end{quote}

\end{fulllineitems}



\chapter{Dataset Module}
\label{\detokenize{dataset:module-dataset}}\label{\detokenize{dataset:dataset-module}}\label{\detokenize{dataset::doc}}\index{module@\spxentry{module}!dataset@\spxentry{dataset}}\index{dataset@\spxentry{dataset}!module@\spxentry{module}}\index{Dataset (class in dataset)@\spxentry{Dataset}\spxextra{class in dataset}}

\begin{fulllineitems}
\phantomsection\label{\detokenize{dataset:dataset.Dataset}}\pysiglinewithargsret{\sphinxbfcode{\sphinxupquote{class }}\sphinxcode{\sphinxupquote{dataset.}}\sphinxbfcode{\sphinxupquote{Dataset}}}{\emph{\DUrole{n}{filename}\DUrole{o}{=}\DUrole{default_value}{None}}, \emph{\DUrole{n}{date\_name}\DUrole{o}{=}\DUrole{default_value}{\textquotesingle{}datetime\textquotesingle{}}}}{}
\sphinxAtStartPar
Class for data retrieval and correlation/causation finding.
\index{EPICS (dataset.Dataset attribute)@\spxentry{EPICS}\spxextra{dataset.Dataset attribute}}

\begin{fulllineitems}
\phantomsection\label{\detokenize{dataset:dataset.Dataset.EPICS}}\pysigline{\sphinxbfcode{\sphinxupquote{EPICS}}}
\sphinxAtStartPar
Flag to indicate whether it is using EPICS or a local csv
\begin{quote}\begin{description}
\item[{Type}] \leavevmode
\sphinxAtStartPar
\sphinxtitleref{bool}

\end{description}\end{quote}

\end{fulllineitems}

\index{dataset (dataset.Dataset attribute)@\spxentry{dataset}\spxextra{dataset.Dataset attribute}}

\begin{fulllineitems}
\phantomsection\label{\detokenize{dataset:dataset.Dataset.dataset}}\pysigline{\sphinxbfcode{\sphinxupquote{dataset}}}
\sphinxAtStartPar
Dataset being used when using local csv
\begin{quote}\begin{description}
\item[{Type}] \leavevmode
\sphinxAtStartPar
\sphinxtitleref{pandas.DataFrame}

\end{description}\end{quote}

\end{fulllineitems}

\index{loop (dataset.Dataset attribute)@\spxentry{loop}\spxextra{dataset.Dataset attribute}}

\begin{fulllineitems}
\phantomsection\label{\detokenize{dataset:dataset.Dataset.loop}}\pysigline{\sphinxbfcode{\sphinxupquote{loop}}}
\sphinxAtStartPar
Loop for async requests
\begin{quote}\begin{description}
\item[{Type}] \leavevmode
\sphinxAtStartPar
\sphinxtitleref{asyncio.loop}

\end{description}\end{quote}

\end{fulllineitems}

\index{last\_searches (dataset.Dataset attribute)@\spxentry{last\_searches}\spxextra{dataset.Dataset attribute}}

\begin{fulllineitems}
\phantomsection\label{\detokenize{dataset:dataset.Dataset.last_searches}}\pysigline{\sphinxbfcode{\sphinxupquote{last\_searches}}}
\sphinxAtStartPar
Dictionary with regex and PV names pairs for all regex already searched
\begin{quote}\begin{description}
\item[{Type}] \leavevmode
\sphinxAtStartPar
\sphinxtitleref{dict}

\end{description}\end{quote}

\end{fulllineitems}

\index{last\_dataset\_metadata (dataset.Dataset attribute)@\spxentry{last\_dataset\_metadata}\spxextra{dataset.Dataset attribute}}

\begin{fulllineitems}
\phantomsection\label{\detokenize{dataset:dataset.Dataset.last_dataset_metadata}}\pysigline{\sphinxbfcode{\sphinxupquote{last\_dataset\_metadata}}}
\sphinxAtStartPar
Dictionary containing the dates and PVs that were last requested
\begin{quote}\begin{description}
\item[{Type}] \leavevmode
\sphinxAtStartPar
\sphinxtitleref{dict}

\end{description}\end{quote}

\end{fulllineitems}

\index{last\_dataset (dataset.Dataset attribute)@\spxentry{last\_dataset}\spxextra{dataset.Dataset attribute}}

\begin{fulllineitems}
\phantomsection\label{\detokenize{dataset:dataset.Dataset.last_dataset}}\pysigline{\sphinxbfcode{\sphinxupquote{last\_dataset}}}
\sphinxAtStartPar
DataFrame with the last requested time series
\begin{quote}\begin{description}
\item[{Type}] \leavevmode
\sphinxAtStartPar
\sphinxtitleref{pandas.DataFrame}

\end{description}\end{quote}

\end{fulllineitems}

\index{causation() (dataset.Dataset method)@\spxentry{causation()}\spxextra{dataset.Dataset method}}

\begin{fulllineitems}
\phantomsection\label{\detokenize{dataset:dataset.Dataset.causation}}\pysiglinewithargsret{\sphinxbfcode{\sphinxupquote{causation}}}{\emph{\DUrole{n}{x\_label}}, \emph{\DUrole{n}{begin}\DUrole{o}{=}\DUrole{default_value}{None}}, \emph{\DUrole{n}{end}\DUrole{o}{=}\DUrole{default_value}{None}}, \emph{\DUrole{n}{margin}\DUrole{o}{=}\DUrole{default_value}{0.2}}, \emph{\DUrole{o}{**}\DUrole{n}{opt}}}{}
\sphinxAtStartPar
Finds causation graph between x\_lab and the PVs in the dataset with TCDF {[}1{]}\_\_
\begin{quote}\begin{description}
\item[{Parameters}] \leavevmode\begin{itemize}
\item {} 
\sphinxAtStartPar
\sphinxstyleliteralstrong{\sphinxupquote{x\_label}} (\sphinxtitleref{str}) \textendash{} Name of the main PV

\item {} 
\sphinxAtStartPar
\sphinxstyleliteralstrong{\sphinxupquote{begin}} (\sphinxtitleref{Datetime.datetime}, optional) \textendash{} Beginning date. Default: None (uses ARCHIVER’s default)

\item {} 
\sphinxAtStartPar
\sphinxstyleliteralstrong{\sphinxupquote{end}} (\sphinxtitleref{Datetime.datetime}, optional) \textendash{} End date. Default: None (uses ARCHIVER’s default)

\item {} 
\sphinxAtStartPar
\sphinxstyleliteralstrong{\sphinxupquote{margin}} (\sphinxtitleref{float}, optional) \textendash{} Percentage of time to consider before and after the defined interval. Default: 0.2 (20\%)

\item {} 
\sphinxAtStartPar
\sphinxstyleliteralstrong{\sphinxupquote{options}} (\sphinxtitleref{tuple(str,int,int,float,int,int,float,int)}, optional) \textendash{} Options for the optimizer, depth, kernel size, significance, stride, log interval, training rate and number of epochs. Default: (‘Adam’,1,4,0.8,4,500,0.01,1000)

\end{itemize}

\item[{Returns}] \leavevmode
\sphinxAtStartPar
Lists containing delays in samples and PV names

\item[{Return type}] \leavevmode
\sphinxAtStartPar
\sphinxtitleref{(List{[}int{]},List{[}str{]})}

\end{description}\end{quote}


\sphinxstrong{See also:}
\nopagebreak

\begin{description}
\item[{\sphinxcode{\sphinxupquote{Causation.get\_causation}}}] \leavevmode
\sphinxAtStartPar
performs TCDF

\end{description}


\subsubsection*{References}

\end{fulllineitems}

\index{causation\_EPICS() (dataset.Dataset method)@\spxentry{causation\_EPICS()}\spxextra{dataset.Dataset method}}

\begin{fulllineitems}
\phantomsection\label{\detokenize{dataset:dataset.Dataset.causation_EPICS}}\pysiglinewithargsret{\sphinxbfcode{\sphinxupquote{causation\_EPICS}}}{\emph{\DUrole{n}{x\_label}}, \emph{\DUrole{n}{regex}}, \emph{\DUrole{n}{begin}\DUrole{o}{=}\DUrole{default_value}{None}}, \emph{\DUrole{n}{end}\DUrole{o}{=}\DUrole{default_value}{None}}, \emph{\DUrole{n}{margin}\DUrole{o}{=}\DUrole{default_value}{0.2}}, \emph{\DUrole{o}{**}\DUrole{n}{opt}}}{}
\sphinxAtStartPar
Finds causation graph between x\_lab and the PVs in the dataset with TCDF {[}2{]}\_\_
\begin{quote}\begin{description}
\item[{Parameters}] \leavevmode\begin{itemize}
\item {} 
\sphinxAtStartPar
\sphinxstyleliteralstrong{\sphinxupquote{x\_label}} (\sphinxtitleref{str}) \textendash{} Name of the main PV

\item {} 
\sphinxAtStartPar
\sphinxstyleliteralstrong{\sphinxupquote{regex}} (\sphinxtitleref{str}) \textendash{} Java regex to match for other PVs

\item {} 
\sphinxAtStartPar
\sphinxstyleliteralstrong{\sphinxupquote{begin}} (\sphinxtitleref{Datetime.datetime}, optional) \textendash{} Beginning date. Default: None (uses ARCHIVER’s default)

\item {} 
\sphinxAtStartPar
\sphinxstyleliteralstrong{\sphinxupquote{end}} (\sphinxtitleref{Datetime.datetime}, optional) \textendash{} End date. Default: None (uses ARCHIVER’s default)

\item {} 
\sphinxAtStartPar
\sphinxstyleliteralstrong{\sphinxupquote{margin}} (\sphinxtitleref{float}, optional) \textendash{} Percentage of time to consider before and after the defined interval. Default: 0.2 (20\%)

\item {} 
\sphinxAtStartPar
\sphinxstyleliteralstrong{\sphinxupquote{options}} (\sphinxtitleref{tuple(str,int,int,float,int,int,float,int)}, optional) \textendash{} Options for the optimizer, depth, kernel size, significance, stride, log interval, training rate and number of epochs. Default: (‘Adam’,1,4,0.8,4,500,0.01,1000)

\end{itemize}

\item[{Returns}] \leavevmode
\sphinxAtStartPar
Lists containing delays in samples and PV names

\item[{Return type}] \leavevmode
\sphinxAtStartPar
\sphinxtitleref{(List{[}int{]},List{[}str{]})}

\end{description}\end{quote}


\sphinxstrong{See also:}
\nopagebreak

\begin{description}
\item[{\sphinxcode{\sphinxupquote{Causation.get\_causation}}}] \leavevmode
\sphinxAtStartPar
performs TCDF

\end{description}


\subsubsection*{References}

\end{fulllineitems}

\index{correlate() (dataset.Dataset method)@\spxentry{correlate()}\spxextra{dataset.Dataset method}}

\begin{fulllineitems}
\phantomsection\label{\detokenize{dataset:dataset.Dataset.correlate}}\pysiglinewithargsret{\sphinxbfcode{\sphinxupquote{correlate}}}{\emph{\DUrole{n}{x\_label}}, \emph{\DUrole{n}{begin}\DUrole{o}{=}\DUrole{default_value}{None}}, \emph{\DUrole{n}{end}\DUrole{o}{=}\DUrole{default_value}{None}}, \emph{\DUrole{n}{margin}\DUrole{o}{=}\DUrole{default_value}{0.2}}, \emph{\DUrole{n}{method}\DUrole{o}{=}\DUrole{default_value}{\textquotesingle{}Pearson\textquotesingle{}}}}{}
\sphinxAtStartPar
Computes the maximum correlation and delay between x\_label and each PV in the csv
\begin{quote}\begin{description}
\item[{Parameters}] \leavevmode\begin{itemize}
\item {} 
\sphinxAtStartPar
\sphinxstyleliteralstrong{\sphinxupquote{x\_label}} (\sphinxtitleref{str}) \textendash{} Name of the main PV

\item {} 
\sphinxAtStartPar
\sphinxstyleliteralstrong{\sphinxupquote{begin}} (\sphinxtitleref{Datetime.datetime}, optional) \textendash{} Beginning date. Default: None (uses ARCHIVER’s default)

\item {} 
\sphinxAtStartPar
\sphinxstyleliteralstrong{\sphinxupquote{end}} (\sphinxtitleref{Datetime.datetime}, optional) \textendash{} End date. Default: None (uses ARCHIVER’s default)

\item {} 
\sphinxAtStartPar
\sphinxstyleliteralstrong{\sphinxupquote{margin}} (\sphinxtitleref{float}, optional) \textendash{} Percentage of time to consider before and after the defined interval. Default: 0.2 (20\%)

\item {} 
\sphinxAtStartPar
\sphinxstyleliteralstrong{\sphinxupquote{method}} (\sphinxtitleref{str}, optional) \textendash{} Method to be used. Default: pearson (other options are spearman, kendall and robust)

\end{itemize}

\item[{Returns}] \leavevmode
\sphinxAtStartPar
Lists containing the correlation coefficients (rounded to 2 decimals), delays in samples and PV names

\item[{Return type}] \leavevmode
\sphinxAtStartPar
\sphinxtitleref{(List{[}float{]},List{[}int{]},List{[}str{]})}

\end{description}\end{quote}

\end{fulllineitems}

\index{correlate\_EPICS() (dataset.Dataset method)@\spxentry{correlate\_EPICS()}\spxextra{dataset.Dataset method}}

\begin{fulllineitems}
\phantomsection\label{\detokenize{dataset:dataset.Dataset.correlate_EPICS}}\pysiglinewithargsret{\sphinxbfcode{\sphinxupquote{correlate\_EPICS}}}{\emph{\DUrole{n}{x\_label}}, \emph{\DUrole{n}{regex}}, \emph{\DUrole{n}{begin}\DUrole{o}{=}\DUrole{default_value}{None}}, \emph{\DUrole{n}{end}\DUrole{o}{=}\DUrole{default_value}{None}}, \emph{\DUrole{n}{margin}\DUrole{o}{=}\DUrole{default_value}{0.2}}, \emph{\DUrole{n}{method}\DUrole{o}{=}\DUrole{default_value}{\textquotesingle{}Pearson\textquotesingle{}}}}{}
\sphinxAtStartPar
Computes the maximum correlation and delay between x\_label and each PV matching regex
\begin{quote}\begin{description}
\item[{Parameters}] \leavevmode\begin{itemize}
\item {} 
\sphinxAtStartPar
\sphinxstyleliteralstrong{\sphinxupquote{x\_label}} (\sphinxtitleref{str}) \textendash{} Name of the main PV

\item {} 
\sphinxAtStartPar
\sphinxstyleliteralstrong{\sphinxupquote{regex}} (\sphinxtitleref{str}) \textendash{} Java regex to match

\item {} 
\sphinxAtStartPar
\sphinxstyleliteralstrong{\sphinxupquote{begin}} (\sphinxtitleref{Datetime.datetime}, optional) \textendash{} Beginning date. Default: None (uses ARCHIVER’s default)

\item {} 
\sphinxAtStartPar
\sphinxstyleliteralstrong{\sphinxupquote{end}} (\sphinxtitleref{Datetime.datetime}, optional) \textendash{} End date. Default: None (uses ARCHIVER’s default)

\item {} 
\sphinxAtStartPar
\sphinxstyleliteralstrong{\sphinxupquote{margin}} (\sphinxtitleref{float}, optional) \textendash{} Percentage of time to consider before and after the defined interval. Default: 0.2 (20\%)

\item {} 
\sphinxAtStartPar
\sphinxstyleliteralstrong{\sphinxupquote{method}} (\sphinxtitleref{str}, optional) \textendash{} Method to be used. Default: pearson (other options are spearman, kendall and robust)

\end{itemize}

\item[{Returns}] \leavevmode
\sphinxAtStartPar
Lists containing the correlation coefficients (rounded to 2 decimals), delays in samples and PV names

\item[{Return type}] \leavevmode
\sphinxAtStartPar
\sphinxtitleref{(List{[}float{]},List{[}int{]},List{[}str{]})}

\end{description}\end{quote}

\end{fulllineitems}

\index{get\_EPICS\_pv() (dataset.Dataset method)@\spxentry{get\_EPICS\_pv()}\spxextra{dataset.Dataset method}}

\begin{fulllineitems}
\phantomsection\label{\detokenize{dataset:dataset.Dataset.get_EPICS_pv}}\pysiglinewithargsret{\sphinxbfcode{\sphinxupquote{get\_EPICS\_pv}}}{\emph{\DUrole{n}{name}}, \emph{\DUrole{n}{start\_time}\DUrole{o}{=}\DUrole{default_value}{None}}, \emph{\DUrole{n}{end\_time}\DUrole{o}{=}\DUrole{default_value}{None}}}{}
\sphinxAtStartPar
Gets DataFrame with the PVs requested
\begin{quote}\begin{description}
\item[{Parameters}] \leavevmode
\sphinxAtStartPar
\sphinxstyleliteralstrong{\sphinxupquote{name}} (\sphinxtitleref{List{[}str{]}}) \textendash{} List with PV names to be requested

\item[{Returns}] \leavevmode
\sphinxAtStartPar
DataFrame where each column is one PVs timeseries

\item[{Return type}] \leavevmode
\sphinxAtStartPar
\sphinxtitleref{pandas.DataFrame}

\end{description}\end{quote}
\subsubsection*{Notes}

\sphinxAtStartPar
If the same names have already been requested, reuses the old result, else it makes another request

\end{fulllineitems}

\index{get\_columns() (dataset.Dataset method)@\spxentry{get\_columns()}\spxextra{dataset.Dataset method}}

\begin{fulllineitems}
\phantomsection\label{\detokenize{dataset:dataset.Dataset.get_columns}}\pysiglinewithargsret{\sphinxbfcode{\sphinxupquote{get\_columns}}}{\emph{\DUrole{n}{regex}\DUrole{o}{=}\DUrole{default_value}{\textquotesingle{}.*\textquotesingle{}}}}{}
\sphinxAtStartPar
Gets PV names in opened dataset (for csv datasets only) according to regex
\begin{quote}\begin{description}
\item[{Parameters}] \leavevmode
\sphinxAtStartPar
\sphinxstyleliteralstrong{\sphinxupquote{regex}} (\sphinxtitleref{str}) \textendash{} Java regex to match

\item[{Returns}] \leavevmode
\sphinxAtStartPar
List containing the names

\item[{Return type}] \leavevmode
\sphinxAtStartPar
\sphinxtitleref{List{[}str{]}}

\end{description}\end{quote}

\end{fulllineitems}

\index{get\_fs() (dataset.Dataset method)@\spxentry{get\_fs()}\spxextra{dataset.Dataset method}}

\begin{fulllineitems}
\phantomsection\label{\detokenize{dataset:dataset.Dataset.get_fs}}\pysiglinewithargsret{\sphinxbfcode{\sphinxupquote{get\_fs}}}{\emph{\DUrole{n}{names}}}{}
\sphinxAtStartPar
Finds the sample rate of the time series being names
\begin{quote}\begin{description}
\item[{Parameters}] \leavevmode
\sphinxAtStartPar
\sphinxstyleliteralstrong{\sphinxupquote{names}} (\sphinxtitleref{List{[}str{]}}) \textendash{} Name of the time series being considered

\item[{Returns}] \leavevmode
\sphinxAtStartPar
List with Timedeltas corresponding to the sampling rate for each name in names

\item[{Return type}] \leavevmode
\sphinxAtStartPar
\sphinxtitleref{List{[}Datetime.Timedelta{]}}

\end{description}\end{quote}

\end{fulllineitems}

\index{number\_of\_vars() (dataset.Dataset method)@\spxentry{number\_of\_vars()}\spxextra{dataset.Dataset method}}

\begin{fulllineitems}
\phantomsection\label{\detokenize{dataset:dataset.Dataset.number_of_vars}}\pysiglinewithargsret{\sphinxbfcode{\sphinxupquote{number\_of\_vars}}}{\emph{\DUrole{n}{regex}}}{}
\sphinxAtStartPar
Gets number of elements that match the regex in Archiver
\begin{quote}\begin{description}
\item[{Parameters}] \leavevmode
\sphinxAtStartPar
\sphinxstyleliteralstrong{\sphinxupquote{regex}} (\sphinxtitleref{str}) \textendash{} Java regex for the PVs being considered

\item[{Returns}] \leavevmode
\sphinxAtStartPar
Number of PVs matching the regex

\item[{Return type}] \leavevmode
\sphinxAtStartPar
int

\end{description}\end{quote}
\subsubsection*{Notes}

\sphinxAtStartPar
If the same expression has already been searched, reuses the old result, else it makes another request

\end{fulllineitems}

\index{to\_date() (dataset.Dataset method)@\spxentry{to\_date()}\spxextra{dataset.Dataset method}}

\begin{fulllineitems}
\phantomsection\label{\detokenize{dataset:dataset.Dataset.to_date}}\pysiglinewithargsret{\sphinxbfcode{\sphinxupquote{to\_date}}}{\emph{\DUrole{n}{delays}}, \emph{\DUrole{n}{names}}}{}
\sphinxAtStartPar
Converts delays from samples to times
\begin{quote}\begin{description}
\item[{Parameters}] \leavevmode\begin{itemize}
\item {} 
\sphinxAtStartPar
\sphinxstyleliteralstrong{\sphinxupquote{delays}} (\sphinxtitleref{List{[}int{]}}) \textendash{} Delays in samples

\item {} 
\sphinxAtStartPar
\sphinxstyleliteralstrong{\sphinxupquote{names}} (\sphinxtitleref{List{[}str{]}}) \textendash{} Name of the time series being considered

\end{itemize}

\item[{Returns}] \leavevmode
\sphinxAtStartPar
List with Timedeltas corresponding to the delays for each name in names

\item[{Return type}] \leavevmode
\sphinxAtStartPar
\sphinxtitleref{List{[}Datetime.Timedelta{]}}

\end{description}\end{quote}

\end{fulllineitems}


\end{fulllineitems}



\chapter{EPICS Requests Module}
\label{\detokenize{epics_requests:module-epics_requests}}\label{\detokenize{epics_requests:epics-requests-module}}\label{\detokenize{epics_requests::doc}}\index{module@\spxentry{module}!epics\_requests@\spxentry{epics\_requests}}\index{epics\_requests@\spxentry{epics\_requests}!module@\spxentry{module}}\index{call\_fetch() (in module epics\_requests)@\spxentry{call\_fetch()}\spxextra{in module epics\_requests}}

\begin{fulllineitems}
\phantomsection\label{\detokenize{epics_requests:epics_requests.call_fetch}}\pysiglinewithargsret{\sphinxbfcode{\sphinxupquote{async }}\sphinxcode{\sphinxupquote{epics\_requests.}}\sphinxbfcode{\sphinxupquote{call\_fetch}}}{\emph{\DUrole{n}{pv\_list}}, \emph{\DUrole{n}{dt\_init}}, \emph{\DUrole{n}{dt\_end}}}{}
\sphinxAtStartPar
Get PVs time series
\begin{quote}\begin{description}
\item[{Parameters}] \leavevmode\begin{itemize}
\item {} 
\sphinxAtStartPar
\sphinxstyleliteralstrong{\sphinxupquote{pv\_list}} (\sphinxtitleref{List{[}str{]}}) \textendash{} List with the name of all PVs to request

\item {} 
\sphinxAtStartPar
\sphinxstyleliteralstrong{\sphinxupquote{dt\_init}} (aware \sphinxtitleref{Datetime.datetime}) \textendash{} Date and time indicating where to begin the timeseries

\item {} 
\sphinxAtStartPar
\sphinxstyleliteralstrong{\sphinxupquote{dt\_end}} (aware \sphinxtitleref{Datetime.datetime}) \textendash{} Date and time indicating where to end the timeseries

\end{itemize}

\item[{Returns}] \leavevmode
\sphinxAtStartPar
DataFrame where the columns correspond to each PV name in pv\_list and the index is a Index with the date as a \sphinxtitleref{str}

\item[{Return type}] \leavevmode
\sphinxAtStartPar
\sphinxtitleref{pandas.DataFrame}

\end{description}\end{quote}

\end{fulllineitems}

\index{correct\_datetime() (in module epics\_requests)@\spxentry{correct\_datetime()}\spxextra{in module epics\_requests}}

\begin{fulllineitems}
\phantomsection\label{\detokenize{epics_requests:epics_requests.correct_datetime}}\pysiglinewithargsret{\sphinxcode{\sphinxupquote{epics\_requests.}}\sphinxbfcode{\sphinxupquote{correct\_datetime}}}{\emph{\DUrole{n}{x}}}{}
\sphinxAtStartPar
Turn Dataframes Index to DatetimeIndex
\begin{quote}\begin{description}
\item[{Parameters}] \leavevmode
\sphinxAtStartPar
\sphinxstyleliteralstrong{\sphinxupquote{x}} (\sphinxtitleref{pandas.DataFrame}) \textendash{} DataFrame with index corresponding to strings of datetimes

\item[{Returns}] \leavevmode
\sphinxAtStartPar
DataFrame with the new DatetimeIndex

\item[{Return type}] \leavevmode
\sphinxAtStartPar
\sphinxtitleref{pandas.DataFrame}

\end{description}\end{quote}

\end{fulllineitems}

\index{get\_names() (in module epics\_requests)@\spxentry{get\_names()}\spxextra{in module epics\_requests}}

\begin{fulllineitems}
\phantomsection\label{\detokenize{epics_requests:epics_requests.get_names}}\pysiglinewithargsret{\sphinxcode{\sphinxupquote{epics\_requests.}}\sphinxbfcode{\sphinxupquote{get\_names}}}{\emph{\DUrole{n}{regex}\DUrole{o}{=}\DUrole{default_value}{None}}, \emph{\DUrole{n}{limit}\DUrole{o}{=}\DUrole{default_value}{100}}}{}
\sphinxAtStartPar
Get all pv names from Archiver that correspond to the given regex
\begin{quote}\begin{description}
\item[{Parameters}] \leavevmode\begin{itemize}
\item {} 
\sphinxAtStartPar
\sphinxstyleliteralstrong{\sphinxupquote{regex}} (\sphinxtitleref{str}, optional) \textendash{} Regex that will be used. Default: None, which matches all pvs (in the order of millions)

\item {} 
\sphinxAtStartPar
\sphinxstyleliteralstrong{\sphinxupquote{limit}} (\sphinxtitleref{int}, optional) \textendash{} Maximum number of names to get (the larger the number, the longer it takes)

\end{itemize}

\item[{Returns}] \leavevmode
\sphinxAtStartPar
List containing all the names

\item[{Return type}] \leavevmode
\sphinxAtStartPar
\sphinxtitleref{List{[}str{]}}

\end{description}\end{quote}

\end{fulllineitems}



\chapter{Layout Module}
\label{\detokenize{layout:module-layout}}\label{\detokenize{layout:layout-module}}\label{\detokenize{layout::doc}}\index{module@\spxentry{module}!layout@\spxentry{layout}}\index{layout@\spxentry{layout}!module@\spxentry{module}}\index{get\_error\_layout() (in module layout)@\spxentry{get\_error\_layout()}\spxextra{in module layout}}

\begin{fulllineitems}
\phantomsection\label{\detokenize{layout:layout.get_error_layout}}\pysiglinewithargsret{\sphinxcode{\sphinxupquote{layout.}}\sphinxbfcode{\sphinxupquote{get\_error\_layout}}}{}{}
\sphinxAtStartPar
Get layout for error message windows
\begin{quote}\begin{description}
\item[{Returns}] \leavevmode
\sphinxAtStartPar
Layout for error message window

\item[{Return type}] \leavevmode
\sphinxAtStartPar
\sphinxtitleref{List{[}List{[}Element{]}{]}}

\end{description}\end{quote}

\end{fulllineitems}

\index{get\_fig\_size() (in module layout)@\spxentry{get\_fig\_size()}\spxextra{in module layout}}

\begin{fulllineitems}
\phantomsection\label{\detokenize{layout:layout.get_fig_size}}\pysiglinewithargsret{\sphinxcode{\sphinxupquote{layout.}}\sphinxbfcode{\sphinxupquote{get\_fig\_size}}}{}{}
\sphinxAtStartPar
Get size of image to fit current window
\begin{quote}\begin{description}
\item[{Returns}] \leavevmode
\sphinxAtStartPar
Width and Height of the image in pixels

\item[{Return type}] \leavevmode
\sphinxAtStartPar
\sphinxtitleref{(int, int)}

\end{description}\end{quote}

\end{fulllineitems}

\index{get\_layout() (in module layout)@\spxentry{get\_layout()}\spxextra{in module layout}}

\begin{fulllineitems}
\phantomsection\label{\detokenize{layout:layout.get_layout}}\pysiglinewithargsret{\sphinxcode{\sphinxupquote{layout.}}\sphinxbfcode{\sphinxupquote{get\_layout}}}{}{}
\sphinxAtStartPar
Get layout for main application window
\begin{quote}\begin{description}
\item[{Returns}] \leavevmode
\sphinxAtStartPar
Layout for main application window

\item[{Return type}] \leavevmode
\sphinxAtStartPar
\sphinxtitleref{List{[}List{[}Element{]}{]}}

\end{description}\end{quote}

\end{fulllineitems}

\index{get\_param\_layout() (in module layout)@\spxentry{get\_param\_layout()}\spxextra{in module layout}}

\begin{fulllineitems}
\phantomsection\label{\detokenize{layout:layout.get_param_layout}}\pysiglinewithargsret{\sphinxcode{\sphinxupquote{layout.}}\sphinxbfcode{\sphinxupquote{get\_param\_layout}}}{}{}
\sphinxAtStartPar
Get layout for causality finding parameters definition window
\begin{quote}\begin{description}
\item[{Returns}] \leavevmode
\sphinxAtStartPar
Layout for causality finding parameters window

\item[{Return type}] \leavevmode
\sphinxAtStartPar
\sphinxtitleref{List{[}List{[}Element{]}{]}}

\end{description}\end{quote}

\end{fulllineitems}

\index{window\_dimension() (in module layout)@\spxentry{window\_dimension()}\spxextra{in module layout}}

\begin{fulllineitems}
\phantomsection\label{\detokenize{layout:layout.window_dimension}}\pysiglinewithargsret{\sphinxcode{\sphinxupquote{layout.}}\sphinxbfcode{\sphinxupquote{window\_dimension}}}{\emph{\DUrole{n}{monospaced\_font}}}{}
\sphinxAtStartPar
Get window dimensions with respect to a font.
\begin{quote}\begin{description}
\item[{Parameters}] \leavevmode
\sphinxAtStartPar
\sphinxstyleliteralstrong{\sphinxupquote{monospaced\_font}} (\sphinxtitleref{tkinter.font.Font}) \textendash{} Font which will be used as unit of measure

\item[{Returns}] \leavevmode
\sphinxAtStartPar
Width and Height of the screen in respective font units

\item[{Return type}] \leavevmode
\sphinxAtStartPar
(int, int)

\end{description}\end{quote}

\end{fulllineitems}



\chapter{TCDF Module}
\label{\detokenize{tcdf:module-tcdf}}\label{\detokenize{tcdf:tcdf-module}}\label{\detokenize{tcdf::doc}}\index{module@\spxentry{module}!tcdf@\spxentry{tcdf}}\index{tcdf@\spxentry{tcdf}!module@\spxentry{module}}\index{findcauses() (in module tcdf)@\spxentry{findcauses()}\spxextra{in module tcdf}}

\begin{fulllineitems}
\phantomsection\label{\detokenize{tcdf:tcdf.findcauses}}\pysiglinewithargsret{\sphinxcode{\sphinxupquote{tcdf.}}\sphinxbfcode{\sphinxupquote{findcauses}}}{\emph{\DUrole{n}{target}}, \emph{\DUrole{n}{cuda}}, \emph{\DUrole{n}{epochs}}, \emph{\DUrole{n}{kernel\_size}}, \emph{\DUrole{n}{layers}}, \emph{\DUrole{n}{log\_interval}}, \emph{\DUrole{n}{lr}}, \emph{\DUrole{n}{optimizername}}, \emph{\DUrole{n}{seed}}, \emph{\DUrole{n}{dilation\_c}}, \emph{\DUrole{n}{significance}}, \emph{\DUrole{n}{data}}}{}
\sphinxAtStartPar
Discovers potential causes of one target time series, validates these potential causes with PIVM and discovers the corresponding time delays

\end{fulllineitems}

\index{preparedata() (in module tcdf)@\spxentry{preparedata()}\spxextra{in module tcdf}}

\begin{fulllineitems}
\phantomsection\label{\detokenize{tcdf:tcdf.preparedata}}\pysiglinewithargsret{\sphinxcode{\sphinxupquote{tcdf.}}\sphinxbfcode{\sphinxupquote{preparedata}}}{\emph{\DUrole{n}{df\_data}}, \emph{\DUrole{n}{target}}}{}
\sphinxAtStartPar
Reads data from csv file and transforms it to two PyTorch tensors: dataset x and target time series y that has to be predicted.

\end{fulllineitems}

\index{train() (in module tcdf)@\spxentry{train()}\spxextra{in module tcdf}}

\begin{fulllineitems}
\phantomsection\label{\detokenize{tcdf:tcdf.train}}\pysiglinewithargsret{\sphinxcode{\sphinxupquote{tcdf.}}\sphinxbfcode{\sphinxupquote{train}}}{\emph{\DUrole{n}{epoch}}, \emph{\DUrole{n}{traindata}}, \emph{\DUrole{n}{traintarget}}, \emph{\DUrole{n}{modelname}}, \emph{\DUrole{n}{optimizer}}, \emph{\DUrole{n}{log\_interval}}, \emph{\DUrole{n}{epochs}}}{}
\sphinxAtStartPar
Trains model by performing one epoch and returns attention scores and loss.

\end{fulllineitems}



\chapter{Plots Module}
\label{\detokenize{plots:module-plots}}\label{\detokenize{plots:plots-module}}\label{\detokenize{plots::doc}}\index{module@\spxentry{module}!plots@\spxentry{plots}}\index{plots@\spxentry{plots}!module@\spxentry{module}}\index{Plots (class in plots)@\spxentry{Plots}\spxextra{class in plots}}

\begin{fulllineitems}
\phantomsection\label{\detokenize{plots:plots.Plots}}\pysiglinewithargsret{\sphinxbfcode{\sphinxupquote{class }}\sphinxcode{\sphinxupquote{plots.}}\sphinxbfcode{\sphinxupquote{Plots}}}{\emph{\DUrole{n}{canvas}}, \emph{\DUrole{n}{FIGSIZE\_X}\DUrole{o}{=}\DUrole{default_value}{800}}, \emph{\DUrole{n}{FIGSIZE\_Y}\DUrole{o}{=}\DUrole{default_value}{800}}}{}
\sphinxAtStartPar
Class responsible for plotting and displaying the figures on the Canvas.
Also controls interactions with the plots.
\index{clear() (plots.Plots method)@\spxentry{clear()}\spxextra{plots.Plots method}}

\begin{fulllineitems}
\phantomsection\label{\detokenize{plots:plots.Plots.clear}}\pysiglinewithargsret{\sphinxbfcode{\sphinxupquote{clear}}}{}{}
\sphinxAtStartPar
Removes the markers from the plot

\end{fulllineitems}

\index{get\_times() (plots.Plots method)@\spxentry{get\_times()}\spxextra{plots.Plots method}}

\begin{fulllineitems}
\phantomsection\label{\detokenize{plots:plots.Plots.get_times}}\pysiglinewithargsret{\sphinxbfcode{\sphinxupquote{get\_times}}}{}{}
\sphinxAtStartPar
Times of the markers

\end{fulllineitems}

\index{on\_click() (plots.Plots method)@\spxentry{on\_click()}\spxextra{plots.Plots method}}

\begin{fulllineitems}
\phantomsection\label{\detokenize{plots:plots.Plots.on_click}}\pysiglinewithargsret{\sphinxbfcode{\sphinxupquote{on\_click}}}{\emph{\DUrole{n}{event}}}{}
\sphinxAtStartPar
Action to be performed when the plot is clicked. Manages the markers in the plot

\end{fulllineitems}

\index{twinx\_canvas() (plots.Plots method)@\spxentry{twinx\_canvas()}\spxextra{plots.Plots method}}

\begin{fulllineitems}
\phantomsection\label{\detokenize{plots:plots.Plots.twinx_canvas}}\pysiglinewithargsret{\sphinxbfcode{\sphinxupquote{twinx\_canvas}}}{\emph{\DUrole{n}{x}}, \emph{\DUrole{n}{x\_label}}, \emph{\DUrole{n}{y}}, \emph{\DUrole{n}{y\_label}}, \emph{\DUrole{n}{colors}\DUrole{o}{=}\DUrole{default_value}{\textquotesingle{}r\textquotesingle{}}}, \emph{\DUrole{n}{t}\DUrole{o}{=}\DUrole{default_value}{None}}, \emph{\DUrole{n}{t\_label}\DUrole{o}{=}\DUrole{default_value}{\textquotesingle{}Time\textquotesingle{}}}}{}
\sphinxAtStartPar
Plot two variables in different y axis
\begin{quote}\begin{description}
\item[{Parameters}] \leavevmode\begin{itemize}
\item {} 
\sphinxAtStartPar
\sphinxstyleliteralstrong{\sphinxupquote{x}} (\sphinxtitleref{pandas.Series}) \textendash{} First time series

\item {} 
\sphinxAtStartPar
\sphinxstyleliteralstrong{\sphinxupquote{x\_label}} (\sphinxtitleref{str}) \textendash{} Name of the first time series (will appear on axis)

\item {} 
\sphinxAtStartPar
\sphinxstyleliteralstrong{\sphinxupquote{y}} (\sphinxtitleref{List{[}pandas.Series{]}}) \textendash{} Secondary time series

\item {} 
\sphinxAtStartPar
\sphinxstyleliteralstrong{\sphinxupquote{y\_label}} (\sphinxtitleref{List{[}str{]}}) \textendash{} Name of the secondary time series (will appear on axis)

\item {} 
\sphinxAtStartPar
\sphinxstyleliteralstrong{\sphinxupquote{colors}} ({\color{red}\bfseries{}\textasciigrave{}}List{[}str{]}, optional) \textendash{} Names of the colors to use for each variable (same as matplotlib names). Default: ‘r’

\item {} 
\sphinxAtStartPar
\sphinxstyleliteralstrong{\sphinxupquote{t}} (\sphinxstyleliteralemphasis{\sphinxupquote{iterable}}\sphinxstyleliteralemphasis{\sphinxupquote{, }}\sphinxstyleliteralemphasis{\sphinxupquote{optional}}) \textendash{} Values for the x axis of the plot. Default: None (uses series index as x)

\item {} 
\sphinxAtStartPar
\sphinxstyleliteralstrong{\sphinxupquote{t\_label}} (\sphinxtitleref{str}) \textendash{} Name of the x axis. Default: ‘Time’

\end{itemize}

\end{description}\end{quote}

\end{fulllineitems}

\index{update\_canvas() (plots.Plots method)@\spxentry{update\_canvas()}\spxextra{plots.Plots method}}

\begin{fulllineitems}
\phantomsection\label{\detokenize{plots:plots.Plots.update_canvas}}\pysiglinewithargsret{\sphinxbfcode{\sphinxupquote{update\_canvas}}}{\emph{\DUrole{n}{x}}, \emph{\DUrole{n}{x\_label}}, \emph{\DUrole{n}{t}\DUrole{o}{=}\DUrole{default_value}{None}}, \emph{\DUrole{n}{t\_label}\DUrole{o}{=}\DUrole{default_value}{\textquotesingle{}Time\textquotesingle{}}}}{}
\sphinxAtStartPar
Plot variable
\begin{quote}\begin{description}
\item[{Parameters}] \leavevmode\begin{itemize}
\item {} 
\sphinxAtStartPar
\sphinxstyleliteralstrong{\sphinxupquote{x}} (\sphinxtitleref{pandas.Series}) \textendash{} Time series

\item {} 
\sphinxAtStartPar
\sphinxstyleliteralstrong{\sphinxupquote{x\_label}} (\sphinxtitleref{str}) \textendash{} Name of the first time series (will appear on axis)

\item {} 
\sphinxAtStartPar
\sphinxstyleliteralstrong{\sphinxupquote{t}} (\sphinxstyleliteralemphasis{\sphinxupquote{iterable}}\sphinxstyleliteralemphasis{\sphinxupquote{, }}\sphinxstyleliteralemphasis{\sphinxupquote{optional}}) \textendash{} Values for the x axis of the plot. Default: None (uses series index as x)

\item {} 
\sphinxAtStartPar
\sphinxstyleliteralstrong{\sphinxupquote{t\_label}} (\sphinxtitleref{str}) \textendash{} Name of the x axis. Default: ‘Time’

\end{itemize}

\end{description}\end{quote}

\end{fulllineitems}


\end{fulllineitems}



\chapter{Indices and tables}
\label{\detokenize{index:indices-and-tables}}\begin{itemize}
\item {} 
\sphinxAtStartPar
\DUrole{xref,std,std-ref}{genindex}

\item {} 
\sphinxAtStartPar
\DUrole{xref,std,std-ref}{modindex}

\item {} 
\sphinxAtStartPar
\DUrole{xref,std,std-ref}{search}

\end{itemize}


\renewcommand{\indexname}{Python Module Index}
\begin{sphinxtheindex}
\let\bigletter\sphinxstyleindexlettergroup
\bigletter{a}
\item\relax\sphinxstyleindexentry{application\_logic}\sphinxstyleindexpageref{application_logic:\detokenize{module-application_logic}}
\indexspace
\bigletter{c}
\item\relax\sphinxstyleindexentry{causations}\sphinxstyleindexpageref{causations:\detokenize{module-causations}}
\item\relax\sphinxstyleindexentry{correlator}\sphinxstyleindexpageref{correlator:\detokenize{module-correlator}}
\indexspace
\bigletter{d}
\item\relax\sphinxstyleindexentry{dataset}\sphinxstyleindexpageref{dataset:\detokenize{module-dataset}}
\indexspace
\bigletter{e}
\item\relax\sphinxstyleindexentry{epics\_requests}\sphinxstyleindexpageref{epics_requests:\detokenize{module-epics_requests}}
\indexspace
\bigletter{i}
\item\relax\sphinxstyleindexentry{interface}\sphinxstyleindexpageref{interface:\detokenize{module-interface}}
\indexspace
\bigletter{l}
\item\relax\sphinxstyleindexentry{layout}\sphinxstyleindexpageref{layout:\detokenize{module-layout}}
\indexspace
\bigletter{p}
\item\relax\sphinxstyleindexentry{plots}\sphinxstyleindexpageref{plots:\detokenize{module-plots}}
\indexspace
\bigletter{t}
\item\relax\sphinxstyleindexentry{tcdf}\sphinxstyleindexpageref{tcdf:\detokenize{module-tcdf}}
\end{sphinxtheindex}

\renewcommand{\indexname}{Index}
\printindex
\end{document}